\section{Definição de enumerabilidade}

\begin{frame}[fragile]{Funções parciais}

    \begin{block}{Funções parciais}
        Sejam $A$ e $B$ conjuntos. A função $f: D\to B$ é uma função \textbf{parcial} de $A$ em 
        $B$ se $D\subset A$ é um subconjunto próprio de $A$. Se $D = A$ a função $f$ é dita uma
        função \textbf{total} de $A$ em $B$.
    \end{block}

\end{frame}

\begin{frame}[fragile]{Enumerabilidade}

    \begin{block}{Conjuntos enumeráveis}
        Um conjunto $A$ é enumerável se, e somente se, ele é imagem de ao menos uma função (total 
        ou parcial) de $\mathbb{N}$ em $A$.
    \end{block}

    \vspace{0.2in}

    \textbf{Observação}: informalmente, um conjunto $A$ é enumerável se seus elementos podem
        ser listados em ordem, isto é, um primeiro elemento, um segundo elemento, etc,
        \[
            a_1, a_2, a_3, \ldots
        \]
        de modo que, cedo ou tarde, todos os elementos de $A$ sejam listados, ao menos, uma vez. 
\end{frame}
