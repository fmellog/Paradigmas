\section{Paradigma Lógico}


\begin{frame}[fragile]{Paradigma Lógico}

    \begin{itemize}
        \item O paradigma lógico foi elaborado a partir de um teorema proposto no contexto
            do processamento de linguagens naturais

        \item Ele permite prototipação e desenvolvimento rápidos devido sua proximidade
            semântica com a especificação lógica do problema ser resolvido

        \item A programação, sob o viés do paradigma lógico, é declarativa

        \item Os programas são compostos por uma série de relações lógicas e consultas, as
            quais questionam se uma determinada proposição é ou não verdadeira

        \item Caso seja verdadeira, também é possível determinar qual atribuição de valores
            lógicos às variáveis da sentença aberta a torna uma proposição verdadeira

        \item Para se tornar um paradigma prático, é necessário o apoio de subrotinas
            extra-lógicas que controlem os processos de entrada e saída de dados e que 
            manipulem o fluxo de execução do programa
    \end{itemize}

\end{frame}

\begin{frame}[fragile]{Prolog}

    \begin{itemize}
        \item A linguagem de programação lógica Prolog foi proposta em 1972 por 
           Alain Colmerauer e Philippe Roussel, tendo como base os trabalhos de Robert Kowalski 

        \item Prolog é uma contração da expressão ``\textit{PROgramming in LOGic}''

        \item Ela tem raízes na lógica de primeira ordem

        \item O SWI Prolog pode ser instalado em distribuições com suporte ao
            \texttt{apt} por meio do comando

            \inputsyntax{bash}{codes/install.sh}

        \item O interpretador (\textit{listener}) Prolog pode ser invocado com o comando

            \inputsyntax{sh}{codes/prolog.sh}

        \item Para validar a instalação, utilize o comando

            \inputsyntax{sh}{codes/version.sh}
    \end{itemize}

\end{frame}

\begin{frame}[fragile]{Programas em Prolog}

    \begin{itemize}

        \item Um programa em Prolog é composto de uma coleção de pequenas unidades modulares, 
            denominadas \textbf{predicados}

        \item Os predicados são semelhantes às subrotinas de outras linguagens

        \item Eles podem ser testados e adicionados separadamente em um programa, de modo que 
            é possível construir programas incrementalmente

        \item O ato de inserir de códigos no interpretador Prolog é denominado
            \textbf{consultar}

        \item É possível consultar diretamente um arquivo-fonte no interpretador Prolog por
            meio do predicado \code{prolog}{consult()}:

            \inputsyntax{prolog}{codes/consult.pl}

    \end{itemize}

\end{frame}

\begin{frame}[fragile]{Programas em Prolog}

    \begin{itemize}
        \item Se o arquivo for modificado, ele deve ser relido através do predicado 
            \code{prolog}{reconsult()}:

            \inputsyntax{prolog}{codes/reconsult.pl}

        \item A extensão dos arquivos-fonte deve ser `\code{prolog}{.pl}'

        \item Usando o comando

            \inputsyntax{sh}{codes/run.sh}

        o conteúdo de \texttt{source.pl} é carregado no interpretador e o terminal fica 
        pronto para consultas

        \item Para fazer uma consulta sem entrar no modo iterativo, use a opção `\texttt{-g}'
            para estabelecer o objetivo e a opção `\texttt{-t}' para encerrar o Prolog:

            \inputsyntax{sh}{codes/bash.sh}

    \end{itemize}

\end{frame}

\begin{frame}[fragile]{Exemplo de programa em Prolog}

    \begin{itemize}
        \item A proposição ``\textit{Todo estudante da FCTE é estudante da UnB}'' pode
            ser declarada em Prolog da seguinte forma:

            \inputsyntax{prolog}{codes/unbfga.pl}

        \item Outra forma de ler esta mesma proposição seria ``\textit{Para todo X, X é aluno 
            da UnB se X estuda na FCTE}''

        \item Afirmações sobre estudantes da FCTE podem ser feitas por meio do predicado
            \code{prolog}{fcte/1}:

            \inputsyntax{prolog}{codes/fga.pl}
    
        \item Observe o ponto final (`\texttt{.}') que encerrada cada proposição/predicado

    \end{itemize}

\end{frame}

\begin{frame}[fragile]{Exemplo de arquivo-fonte em Prolog}

    \inputsnippet{prolog}{1}{6}{codes/students.pl}

\end{frame}


\begin{frame}[fragile]{Exemplo de programa em Prolog}

    \begin{itemize}
        \item Se estes predicados forem inseridos em um arquivo denominado 
            `\texttt{students.pl}', ele pode ser consultado no interpretador Prolog
            através do predicado \code{prolog}{consult/1}:

            \inputsyntax{prolog}{codes/students_1.pl}

        \item Para saber checar se Ana é estudante da UnB, basta utilizar a consulta

            \inputsyntax{prolog}{codes/students_2.pl}

        \item Para saber se Diana também é estudante da UnB, a consulta deve ser 

            \inputsyntax{prolog}{codes/students_3.pl}
    \end{itemize}

\end{frame}

\begin{frame}[fragile]{Exemplo de programa em Prolog}

    \begin{itemize}
        \item Para listar todos os estudantes conhecidos da UnB, faça a consulta

            \inputsyntax{prolog}{codes/students_4.pl}

        \item Observe que a consulta atribuiu corretamente \code{prolog}{X = ana}, pois
            Ana é estudante da UnB

        \item A consulta retorna o primeiro valor encontrado para \code{prolog}{X} que
            torna a sentença aberta \code{prolog}{unb(X)} verdadeira

        \item Para obter os demais valores de \code{prolog}{X} que também tornam tal sentença
            verdadeira, utilize o operador ponto-e-vírgula (`\texttt{;}') após cada retorno:

            \inputsyntax{prolog}{codes/students_5.pl}
    \end{itemize}

\end{frame}

\begin{frame}[fragile]{Exemplo de programa em Prolog}

    \begin{itemize}
        \item É possível formatar a lista de todos os estudantes da UnB por meio dos 
            predicados \code{prolog}{write/1} e \code{prolog}{nl/0}

        \item O predicado \code{prolog}{write/1} escreve seu argumento no terminal

        \item O predicado \code{prolog}{nl/0} inicia uma nova linha no terminal

        \item Estes predicados podem ser combinados com a consulta sobre os estudantes da
            UnB para formar o predicado \code{prolog}{unb_report/0}:

            \inputsyntax{prolog}{codes/unbreport.pl}

        \item Este novo predicado por ser consultado no interpretador da seguinte forma:

            \inputsyntax{prolog}{codes/unbreport_call.pl}

    \end{itemize}

\end{frame}

\begin{frame}[fragile]{Arquivo-fonte completo do exemplo}

    \inputsnippet{prolog}{1}{21}{codes/students.pl}

\end{frame}

\begin{frame}[fragile]{Terminologia}

    \begin{itemize}
        \item O jargão de Prolog é composto por termos de programação, termos de bancos de 
            dados e termos lógicos

        \item Não há uma divisão clara, em Prolog, entre dados e procedimentos

        \item Um programa em Prolog é um banco de dados Prolog

        \item Um programa é composto por predicados (procedimentos, registros, relações)

        \item Cada predicado é definido por um nome e por um número (\textbf{aridade})

        \item A aridade é o número de argumentos (atributos, campos) do predicado

        \item Dois predicados com nomes iguais, mas aridades distintas, são considerados 
            distintos
    \end{itemize}

\end{frame}

\begin{frame}[fragile]{Terminologia}

    \begin{itemize}
        \item No exemplo anterior são três os predicados: \code{prolog}{unb/1},
            \code{prolog}{unb_report/0} e \code{prolog}{fcte/1}

        \item \code{prolog}{fcte/1} lembra um registro com um campo de outras linguagens

        \item \code{prolog}{unb_report/0} se assemelha a uma subrotina sem argumentos

        \item \code{prolog}{unb/1} remete a uma regra ou proposição, e está em algum lugar 
            entre dado e procedimento

        \item Cada predicado do programa é definido pela existência de uma ou mais cláusulas 
            no banco de dados

        \item No exemplo, \code{prolog}{fcte/1} tem 3 cláusulas, os demais predicados apenas 
            uma cláusula

        \item Cada cláusula pode ser uma \textbf{regra} ou um \textbf{fato}

        \item As três cláusulas de \code{prolog}{fcte/1} são fatos

        \item As demais são regras
    \end{itemize}

\end{frame}


