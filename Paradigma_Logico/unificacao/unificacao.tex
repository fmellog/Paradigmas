\section{Unificação}

\begin{frame}[fragile]{Processos de unificação}

    \begin{table}
        \centering

        \begin{tabularx}{0.9\textwidth}{cX}
            \toprule
            \textbf{Elementos} & \textbf{Comportamento} \\            
            \midrule

            variável e qualquer termo & a variável é unificada e atada com qualquer termo, 
                inclusive outras variáveis \\
            \midrule

            primitivo e primitivo & dois termos primitivos (átomos ou inteiros) só se unificam 
                se ambos são idênticos \\
            \midrule

            estrutura e estrutura & se unificam se tem mesmo construtor, mesma aridade e cada 
                par de argumentos correspondentes se unificam \\
            \bottomrule
        \end{tabularx}
    \end{table}

\end{frame}

\begin{frame}[fragile]{Igualdade}

    \begin{itemize}
        \item O predicado pré-definido \code{prolog}{=/2} é bem sucedido se ambos argumentos 
            se unificam e falha, caso contrário

            \inputsyntax{prolog}{codes/equals.pl}

    \end{itemize}

\end{frame}

\begin{frame}[fragile]{Variáveis e unificação}

    \begin{itemize}
        \item Se uma variável se unificar com uma primitiva, ela assume seu valor

            \inputsyntax{prolog}{codes/variables.pl}

    \end{itemize}

\end{frame}

\begin{frame}[fragile]{Variáveis e unificação}

    \begin{itemize}
        \item Variáveis também se unificam

            \inputsyntax{prolog}{codes/variables2.pl}

    \end{itemize}

\end{frame}

\begin{frame}[fragile]{Variáveis e unificação}

    \begin{itemize}
        \item A variável anônima é um \textit{wildcard} e não ata a valor algum

        \item O predicado \code{prolog}{\=/2} unifica apenas se os valores são distintos

            \inputsyntax{prolog}{codes/variables3.pl}
    \end{itemize}

\end{frame}
