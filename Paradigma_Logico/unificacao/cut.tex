\section{\it Cut}

\begin{frame}[fragile]{Predicado {\tt cut}}

    \begin{itemize}
        \item O predicado \textit{cut} (\code{prolog}{!}, ponto de exclamação) permite 
            a interrupção o processo de \textit{backtracking}

        \item Ele congela todas as decisões tomadas pelo \textit{backtracking} até o momento

        \item O principal motivo para o uso deste operador é o aumento de performance, pois
            habilita a poda no \textit{backtracking}

        \item Ele é motivo de controvérsia entre os puristas do paradigma lógico e os 
            pragmáticos

        \item Ele é considerado o \textbf{goto} do paradigma


    \end{itemize}

\end{frame}

\begin{frame}[fragile]{Negação}

    \begin{itemize}
        \item O predicado \code{prolog}{not/1} é implementado em termos do operador 
            \textit{cut} e do predicado \code{prolog}{call/1}, o qual invoca um predicado:

            \inputsyntax{prolog}{codes/not.pl}

        \item Observe que se a chamada do predicado \code{prolog}{X} for bem sucedida, ela
            seguirá para a direita, via porta \code{prolog}{exit}

        \item Daí a execução encontrará o \textit{cut}, interrompendo o \textit{backtracking}

        \item Por fim o predicado \code{prolog}{fail/0} é encontrado, determinando que a 
            consulta retorne falso

        \item Por outro lado, caso a chamada de \code{prolog}{X} falhe, será avaliada a 
            segunda cláusula, a qual sempre é verdadeira

        \item Assim, o predicado \code{prolog}{not/1} inverte o resultado da consulta
            \code{prolog}{X}
    \end{itemize}

\end{frame}
